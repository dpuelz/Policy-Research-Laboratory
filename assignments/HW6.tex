\documentclass[11pt]{article}

% Preamble!!
\usepackage{titling}
\setlength{\voffset}{0.7in}
\setlength{\droptitle}{-10em}
\usepackage{titlesec}
\titlelabel{\thetitle.\quad}
\usepackage{xcolor}
\usepackage{adjustbox}
\usepackage{amsmath, amsthm, amsfonts, amssymb}
\usepackage{graphicx}
\usepackage{setspace}
\usepackage{longtable}
\usepackage{breqn}
\usepackage{lscape}
\usepackage{indentfirst}
\usepackage[labelsep=period,justification=justified,singlelinecheck=false,font = footnotesize, labelfont=bf]{caption}
\usepackage{booktabs}
\usepackage{tabularx,ragged2e}
\usepackage{natbib}
\usepackage{rotating}
\usepackage{placeins}
\usepackage{subcaption}
\usepackage{hyperref}
\definecolor{burntorange}{rgb}{0.8, 0.33, 0.0}
\hypersetup{
    colorlinks=true,
    linkcolor=orange,
    filecolor=magenta,      
    urlcolor=burntorange,
}
\pagenumbering{arabic}

\usepackage{bbm}
\usepackage[margin=1in]{geometry}

\usepackage{enumerate}
\usepackage{array}
\usepackage[T1]{fontenc}
\usepackage[font=small,labelfont=bf,tableposition=top]{caption}
\usepackage{mathtools}
\newcommand\eho{\stackrel{\mathclap{\small\mbox{$H_0$}}}{=}}
\newcommand\sho{\stackrel{\mathclap{\small\mbox{$*$}}}{=}}
\newcommand\dho{\stackrel{\mathclap{\small\mbox{$d$}}}{=}}
\newcommand\qho{\stackrel{\mathclap{\small\mbox{$?$}}}{=}}

% \usepackage{authblk}


\DeclareCaptionLabelFormat{andtable}{#1~#2  \&  \tablename~\thetable}

\usepackage{fullpage, amsmath, amssymb, amsthm, bbm, color}
\usepackage{graphicx,caption,subcaption,placeins}
\usepackage{dsfont}

\usepackage{natbib}
\bibpunct{(}{)}{;}{a}{,}{,}

\newtheorem{assumption}{Assumption}
\newtheorem{definition}{Definition}
\newtheorem{theorem}{Theorem}
\newtheorem{example}{Example}
\newtheorem{procedure}{Procedure}


\usepackage{tikz}
\usetikzlibrary{positioning,chains,fit,shapes,calc}

\definecolor{myblue}{RGB}{80,80,160}
\definecolor{mygreen}{RGB}{80,160,80}


\begin{document}

\title{FIN 373 Homework 6 \\ {\large due: \textbf{10/5/21}}}
\date{}
\maketitle

\vspace{-20mm}

\noindent Instructions: Please submit solutions on canvas.  Only a knitted pdf of an {\tt Rmarkdown} file will be accepted.
\\

\noindent \textbf{Problem 1:} In class and our weekly readings, we encountered an important programming method called the \textit{loop}.
In this exercise, we practice using loops 
with data on the ideological positions of United States
Supreme Court Justices.  Just like legislators, justices make voting decisions 
that we can use to estimate their ideological positions.\footnote{This exercise is based in part on Andrew Martin and Kevin Quinn (2002) ``Dynamic
  Ideal Point Estimation via Markov Chain Monte Carlo for the
  U.S. Supreme Court, 1953-1999.'' \textit{Political Analysis}, 10:2, pp.134-154.}
  
  
The file {\tt justices.csv} contains the following variables:

\vspace{1mm}
\begin{center}
\begin{tabular}{l p{8cm}}
 \hline
\textit{Variable} & \textit{Description} \\
\hline
{\tt term} &               Supreme Court term (a year) \\
{\tt justice} &            Justice's name \\
{\tt idealpt} &           Justice's estimated ideal point in that term \\
{\tt pparty} &             Political party of the president in that term \\
{\tt pres} &               President's name \\
\hline
\end{tabular}
\end{center}

The ideal points of the justices are negative to indicate liberal 
preferences and positive to indicate conservative preferences. 

\begin{enumerate}[a.]
\item We wish to know the median ideal point for the Court during each
  term included in the dataset. First, calculate the median ideal point
  for each term of the Court. Next, generate a plot
  with term on the horizontal axis and ideal point on the vertical axis. 
  Include a dashed horizontal line at zero to indicate a ``neutral''
  ideal point. Be sure to include informative axis labels
  and a plot title.
\item Next, we wish to identify the name of the justice
  with the median ideal point \textit{for each term}. Which justice had the median
  ideal point in the most (potentially
  nonconsecutive) terms? How long did this justice serve on the Court overall?
  What was this justice's average ideal point over his/her entire
  tenure on the Court?
\item  We now turn to the relationship between Supreme Court
  ideology and the president. Specifically, we want to see how
  the ideology of the Supreme Court changes over the course of each president's
  time in office. Begin by creating two empty
  ``container'' vectors: one to hold Democratic presidents, 
  and another for Republican presidents. Label each vector with the presidents' names.
\item Next, for each Democratic president, calculate the shift in Supreme
  Court ideology by subtracting the Court's median ideal point in the 
  president's first term from its median ideal point in the president's last
  term. Use a loop to store these values in your Democratic container vector. 
  Repeat the same process for Republican presidents.
\item What was the mean and standard deviation of the Supreme Court
  ideology shifts you just calculated when looking only at the 
  Democratic presidencies? What about the Republican presidencies?
  Which Republican president's tenure had the largest conservative 
  (positive) shift on the Court? Which Democratic president's tenure
  had the largest liberal (negative) shift? 
\item Create a plot that shows the median Supreme Court ideal point
  over time.  Then, add lines for the ideal points of each unique justice
  to the same plot. The color of each line should be red if the 
  justice was appointed by a Republican and blue if he or she was appointed
  by a Democrat. (You can assume that when a Justice first appears in the 
  data, they were appointed by the president sitting during that term.)
  Label each line with the justice's last name.  Briefly comment on the
  resulting plot.  
\end{enumerate}


\vspace{7mm}
\noindent \textbf{Problem 2:} Across industrialized countries, it is a well-studied phenomenon 
that childless women are paid more on average than mothers. In this exercise, 
we use survey data to investigate how the structural aspects of jobs affect 
the wages of mothers relative to the wages of childless women.\footnote{The exercise is based on: Wei-hsin Yu and J
anet Chen-Lan Kuo (2017) \href{https://doi.org/10.1177/0003122417712729}{``The Motherhood Wage Penalty by Work 
Conditions: How Do Occupational Characteristics Hinder or 
Empower Mothers?''}
\textit{American Sociological Review} 82(4): 744-769.}

In this paper, the authors examine the association between 
the so-called \textit{mother wage penalty} (i.e., the pay gap between mothers 
and non-mothers) and occupational characteristics. Three prominent 
explanations for the motherhood wage penalty--``stressing work-family conflict 
and job performance,'' ``compensating differentials,'' and ``employer 
discrimination''--provide hypotheses about the relationship 
between penalty changes and occupational characteristics.
The authors use data from 16 waves of the National Longitudinal 
Survey of Youth to estimate the effects of five occupational 
characteristics on the mother wage penalty and to test these 
hypotheses. 

This paper uses a type of data known as ``panel data.'' Panel data consist of 
observations on the same people over time. In this example, we are going to 
analyze the same women over multiple years. When analyzing panel data, each 
time period is referred to as a \textit{wave}, so here each year is a wave. 
{\bf The most general form of model for working with panel data is the 
\textit{two-way fixed effects model}, in which there is a fixed effect for each 
woman and for each wave.}

The data file is {\tt yu2017sample.csv}. The names 
and descriptions of variables are:


\vspace{1mm}
\begin{center}
\begin{tabular}{l p{13cm}}
 \hline
\textit{Variable} & \textit{Description} \\
\hline
{\tt PUBID} &            ID of woman \\
{\tt year} &               Year of observation \\
{\tt wage} &               Hourly wage, in cents \\
{\tt numChildren} &        Number of children that the woman has (in this wave) \\
{\tt age} &                Age in years \\
{\tt region} &             Name of region (North East = 1, North Central = 2, South = 3, West = 4) \\
{\tt urban} &              Geographical classification (urban = 1, otherwise = 0) \\     
{\tt marstat} &            Marital status \\
{\tt educ} &               Level of education \\
{\tt school} &             School enrollment (enrolled = {\tt TRUE}, otherwise = {\tt FALSE}) \\
{\tt experience} &         Experience since 14 years old, in days \\
{\tt tenure} &            Current job tenure, in years \\
{\tt tenure2} &            Current job tenure in years, squared \\
 \hline
\end{tabular}
\end{center}

\begin{center}
\begin{tabular}{l p{13cm}}
 \hline
\textit{Variable} & \textit{Description} \\
\hline
{\tt fullTime}   &        Employment status (employed full-time = {\tt TRUE}, otherwise = {\tt FALSE}) \\
{\tt firmSize} &           Size of the firm \\
{\tt multipleLocations} &   Multiple locations indicator (firm with multiple locations = 1, otherwise = 0) \\
{\tt unionized} &          Job unionization status (job is unionized = 1, otherwise = 1) \\
{\tt industry} &           Job's industry type \\
{\tt hazardous} &          Hazard measure for the job (between 1 and 5) \\
{\tt regularity} &         Regularity measure for the job (between 1 and 5) \\
 \hline
\end{tabular}
\end{center}

\begin{enumerate}[a.]
\item How many different women are in the data? How many observations 
  per year? We will refer to each row as a ``person-year observation'' 
  since the row contains data on a given person in a particular year.
  In a few sentences, describe one advantage and one disadvantage of using 
  a contemporary cohort of women rather than an older cohort in 
  estimating the predictors of the mother wage gap.
\item {\tt numChildren} is the variable representing how many children 
  the woman had at the time of an observation. Please provide a table 
  that shows the proportion of observations by the number of children.
  Provide a brief substantive interpretation of the results.
\item Create a new indicator variable {\tt isMother} that takes a value of 1 
  if the woman has at least one child in that year and a value 
  of 0 otherwise. Tabulate the new variable. Briefly comment on 
  the results.
\item Create a new variable called {\tt logwage} that is the log of {\tt wage}. 
  Make two boxplots, one for {\tt wage} and the other for {\tt logwage}, 
  as a function of educational level ({\tt educ}). Compare the two boxplots 
  and discuss the purpose of the log transformation. 
\item In the same graph, plot the mean {\tt logwage} against year for mothers, 
  then for non-mothers in a different color or line type. Include 
  a legend and a proper title. Make sure the figure and axes are 
  clearly labeled. Give a brief interpretation of the results.
\item Run a regression using fixed effects for both {\it woman} and {\it year}. 
You should be sure to include variables for number of children ({\tt numChildren})
and job characteristics ({\tt fullTime}, {\tt firmSize}, {\tt multipleLocations}, 
{\tt unionized}, {\tt industry}). Note: that you should \textit{not} use the {\tt isMother} 
variable you created earlier in this model. Report the coefficient of {\tt numChildren}. 
Provide a brief substantive interpretation of this coefficient and
the coefficients for any two other variables. (\textit{Hint}: fixed effects means including the relevant factor variables in the regression model -- see the bolded statement in the problem introduction).
\item Add interactions between {\tt numChildren} and {\tt regularity} and between 
  {\tt numChildren} and {\tt hazardous} to the model in the previous question. Report the 
  five coefficients involving these variables. Interpret the interaction 
  term for {\tt numChildren} and {\tt hazardous}. Can we interpret 
  the effect of occupation characteristics on motherhood wage penalty 
  as causal? Why or why not?
\end{enumerate}


\end{document}

