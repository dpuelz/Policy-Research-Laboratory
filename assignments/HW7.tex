\documentclass[11pt]{article}

% Preamble!!
\usepackage{titling}
\setlength{\voffset}{0.7in}
\setlength{\droptitle}{-10em}
\usepackage{titlesec}
\titlelabel{\thetitle.\quad}
\usepackage{xcolor}
\usepackage{adjustbox}
\usepackage{amsmath, amsthm, amsfonts, amssymb}
\usepackage{graphicx}
\usepackage{setspace}
\usepackage{longtable}
\usepackage{breqn}
\usepackage{lscape}
\usepackage{indentfirst}
\usepackage[labelsep=period,justification=justified,singlelinecheck=false,font = footnotesize, labelfont=bf]{caption}
\usepackage{booktabs}
\usepackage{tabularx,ragged2e}
\usepackage{natbib}
\usepackage{rotating}
\usepackage{placeins}
\usepackage{subcaption}
\usepackage{hyperref}
\definecolor{burntorange}{rgb}{0.8, 0.33, 0.0}
\hypersetup{
    colorlinks=true,
    linkcolor=orange,
    filecolor=magenta,      
    urlcolor=burntorange,
}
\pagenumbering{arabic}

\usepackage{bbm}
\usepackage[margin=1in]{geometry}

\usepackage{enumerate}
\usepackage{array}
\usepackage[T1]{fontenc}
\usepackage[font=small,labelfont=bf,tableposition=top]{caption}
\usepackage{mathtools}
\newcommand\eho{\stackrel{\mathclap{\small\mbox{$H_0$}}}{=}}
\newcommand\sho{\stackrel{\mathclap{\small\mbox{$*$}}}{=}}
\newcommand\dho{\stackrel{\mathclap{\small\mbox{$d$}}}{=}}
\newcommand\qho{\stackrel{\mathclap{\small\mbox{$?$}}}{=}}

% \usepackage{authblk}


\DeclareCaptionLabelFormat{andtable}{#1~#2  \&  \tablename~\thetable}

\usepackage{fullpage, amsmath, amssymb, amsthm, bbm, color}
\usepackage{graphicx,caption,subcaption,placeins}
\usepackage{dsfont}

\usepackage{natbib}
\bibpunct{(}{)}{;}{a}{,}{,}

\newtheorem{assumption}{Assumption}
\newtheorem{definition}{Definition}
\newtheorem{theorem}{Theorem}
\newtheorem{example}{Example}
\newtheorem{procedure}{Procedure}


\usepackage{tikz}
\usetikzlibrary{positioning,chains,fit,shapes,calc}

\definecolor{myblue}{RGB}{80,80,160}
\definecolor{mygreen}{RGB}{80,160,80}


\begin{document}

\title{FIN 373 Homework 7 \\ {\large due: \textbf{10/12/21}}}
\date{}
\maketitle

\vspace{-20mm}

\noindent Instructions: Please submit solutions on canvas.  Only a knitted pdf of an {\tt Rmarkdown} file will be accepted.
\\

\noindent \textbf{Problem 1:} Financial literacy -- the degree to which an individual is educated in budgeting, life-cycle economics, and basic finance -- is typically thought to be a key predictor of future economic success and happiness.  To date, books on financial wellness are almost as ubiquitous as those on diet and weight loss.\footnote{This exercise is based on the working paper: \href{https://papers.ssrn.com/sol3/papers.cfm?abstract_id=3302978}{Financial Literacy and Perceived Economic Outcomes (2021)}.}

The \href{https://gflec.org/initiatives/national-financial-capability-study/}{National Financial Capability Study} (NFCS) is a large-scale survey that aims to measure financial capability across the United States.  It was first administered in 2009 and every three years thereafter.  The data from 2015 and 2018 are combined in the following csv: {\tt finlit15and18.csv}.  There are several variables, many of which are described in Table 4 of the footnoted paper above.  See also the \href{https://github.com/dpuelz/Policy-Research-Laboratory/blob/main/readings/2018NFCScodebook.pdf}{2018 codebook} for the survey for detail about each variables values.  We will be focusing on the following subset below:

\vspace{1mm}
\begin{center}
\begin{tabular}{l p{13. cm}}
 \hline
\textit{Variable} & \textit{Description} \\
\hline
{\tt Y} &               Perceived economic condition, positive and negative values of {\tt Y} are good and bad, respectively \\
{\tt literacy} &           Measure of financial literacy ranging from 0-6, 6 is most literate \\
{\tt A5_2015} &           Level of education \\
{\tt A3A} &             Age \\
{\tt J2} &              When thinking of your financial investments, how willing are you to take risks?  \\
{\tt A3} &     Gender           \\
{\tt A8} &         Household's income       \\
{\tt E15} &       How many times have you been late with your mortgage payment?         \\
{\tt E20} &           Do you owe more on your home than it is worth?     \\
{\tt F2_2} &          Over the past 12 months, have you carried a balance and were charged interest?      \\
{\tt F2_3} &         Over the past 12 months, in some months I paid the minimum payment only       \\
{\tt F2_4} &        Over the past 12 months, I incurred credit card late fee        \\
{\tt F2_5} &          Over the past 12 months, I was charged an over-the-limit fee for exceeding my credit line      \\
{\tt F2_6} &        Over the past 12 months, I used my card for a cash advance        \\
\hline
\end{tabular}
\end{center}    
  
\begin{enumerate}[a.] 
\item First, how many observations are in the data? Second, describe the survey respondents across some of the variables.  What is the gender and age breakdown?  What is the distribution of respondents across household income (remember that household income, although coded numerically, is a categorical variable.  Be sure to reference the \href{https://github.com/dpuelz/Policy-Research-Laboratory/blob/main/readings/2018NFCScodebook.pdf}{codebook} for category descriptions)?  
\item Compute the average {\tt literacy} difference between females and males.  Is this significantly different from zero (use the bootstrap to characterize the sampling distribution of this difference)?  Provide a 95\% confidence interval around the estimate of this difference.  Treating {\tt J2} as a numerical variable, conduct the same analysis for the average difference in {\tt J2} between genders.  What conclusions do you draw from these results?
\item Fit a simple linear regression model of {\tt literacy} on gender ({\tt A3}).  Report the coefficient and standard error on the gender variable.  Run a bootstrap of this coefficient to characterize the sampling distribution.  How does the standard deviation of the sampling distribution compare to the standard error from the regression output?  How does the coefficient and sampling distribution spread compare to the results from the first part of (b)?  Remember that in order to regress an outcome {\tt Y} on a factor variable {\tt X} (binary or categorical), you need to specify that the covariate is a factor in {\tt R}.  The {\tt R} code to do this is: {\tt fit = lm(Y $\sim$ factor(X))}.
\item Investigate the effect of financial literacy on perceived economic condition.  One approach is to start with relatively small regression models (with a couple variables) and move to a large model with all 12 variables listed above.  How well do your models describe the variation in perceived economic condition?  Does the literacy effect change across models?  If so, how and why?
\end{enumerate}

\vspace{7mm}
\noindent \textbf{Problem 2:} This problem develops a nice causal inference approach using linear regression called \textbf{regression discontinuity}.  Section 4.3.4 in QSS will be helpful here.   


In this exercise, we estimate the effects of increased government
spending on educational attainment, literacy, and poverty
rates.\footnote{This exercise is based on
Litschig, Stephan, and Kevin M Morrison (2013). \href{http://dx.doi.org/10.1257/app.5.4.206}{The Impact of Intergovernmental Transfers on Education Outcomes and Poverty Reduction.}\textit{ American Economic Journal: Applied Economics} 5(4): 206-40. }

Some
scholars argue that government spending accomplishes very little in
environments of high corruption and inequality. Others suggest that in
such environments, accountability pressures and the large demand for
public goods will drive elites to respond.  To address this debate, we
exploit the fact that until 1991, the formula for government transfers
to individual Brazilian municipalities was determined in part by the
municipality's population. This meant that municipalities with
populations below the official cutoff did not receive additional
revenue, while states above the cutoff did.  The data set
{\tt transfer.csv} contains the variables:


\vspace{1mm}
\begin{center}
\begin{tabular}{l p{7 cm}}
 \hline
\textit{Variable} & \textit{Description} \\
\hline
{\tt pop82}     &        Population in 1982 \\
{\tt poverty80} &         Poverty rate of state in 1980 \\
{\tt poverty91} &         Poverty rate of state in 1991 \\
{\tt educ80} &            Average years education of state in 1980 \\
{\tt educ91} &            Average years education of state in 1991 \\
{\tt literate91} &        Literacy rate of state in 1991 \\ 
{\tt state} &             State \\
{\tt region} &            Region \\
{\tt id} &                Municipal ID \\
{\tt year} &              Year of measurement \\
 \hline
\end{tabular}
\end{center} 

\begin{enumerate}[a.]
	\item Begin by creating a variable that determines how close each
  municipality was to the cutoff that determined whether states
  received a transfer or not. Transfers occurred at three separate
  population cutoffs: 10,188, 13,584, and 16,980. Using these cutoffs,
  create a single variable that characterizes the difference from the
  closest population cutoff.  Following the original analysis,
  standardize this measure by dividing the difference with the
  corresponding cutoff and multiply it by 100.  This will yield a
  normalized percent score for the difference between the population
  of each state and the cutoff relative to the cutoff value.
  	\item Subset the data to include
  only those municipalities within 3 points of the funding cutoff on
  either side.  Using regressions, estimate the average causal effect
  of government transfer on each of the three outcome variables of
  interest: educational attainment, literacy, and poverty.  Give a
  brief substantive interpretation of the results.
  \item Visualize the analysis done in the previous question by plotting
  data points, fitted regression lines, and the population threshold.
  Briefly comment on the plot.
  \item Instead of fitting linear regression models, we compute the
  difference in means of the outcome variables between the groups of
  observations above the threshold and below it.  How do the estimates
  differ from what you obtained in the earlier Question?  Is
  the assumption invoked here identical to the one required for the
  analysis conducted there?  Which
  estimates are more appropriate?  Please discuss.
\end{enumerate}














\end{document}

