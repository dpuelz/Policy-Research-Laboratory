\documentclass[11pt]{article}

% Preamble!!
\usepackage{titling}
\setlength{\voffset}{0.7in}
\setlength{\droptitle}{-10em}
\usepackage{titlesec}
\titlelabel{\thetitle.\quad}
\usepackage{xcolor}
\usepackage{adjustbox}
\usepackage{amsmath, amsthm, amsfonts, amssymb}
\usepackage{graphicx}
\usepackage{setspace}
\usepackage{longtable}
\usepackage{breqn}
\usepackage{lscape}
\usepackage{indentfirst}
\usepackage[labelsep=period,justification=justified,singlelinecheck=false,font = footnotesize, labelfont=bf]{caption}
\usepackage{booktabs}
\usepackage{tabularx,ragged2e}
\usepackage{natbib}
\usepackage{rotating}
\usepackage{placeins}
\usepackage{subcaption}
\usepackage{hyperref}
\definecolor{burntorange}{rgb}{0.8, 0.33, 0.0}
\hypersetup{
    colorlinks=true,
    linkcolor=orange,
    filecolor=magenta,      
    urlcolor=burntorange,
}
\pagenumbering{arabic}

\usepackage{bbm}
\usepackage[margin=1in]{geometry}

\usepackage{enumerate}
\usepackage{array}
\usepackage[T1]{fontenc}
\usepackage[font=small,labelfont=bf,tableposition=top]{caption}
\usepackage{mathtools}
\newcommand\eho{\stackrel{\mathclap{\small\mbox{$H_0$}}}{=}}
\newcommand\sho{\stackrel{\mathclap{\small\mbox{$*$}}}{=}}
\newcommand\dho{\stackrel{\mathclap{\small\mbox{$d$}}}{=}}
\newcommand\qho{\stackrel{\mathclap{\small\mbox{$?$}}}{=}}

% \usepackage{authblk}


\DeclareCaptionLabelFormat{andtable}{#1~#2  \&  \tablename~\thetable}

\usepackage{fullpage, amsmath, amssymb, amsthm, bbm, color}
\usepackage{graphicx,caption,subcaption,placeins}
\usepackage{dsfont}

\usepackage{natbib}
\bibpunct{(}{)}{;}{a}{,}{,}

\newtheorem{assumption}{Assumption}
\newtheorem{definition}{Definition}
\newtheorem{theorem}{Theorem}
\newtheorem{example}{Example}
\newtheorem{procedure}{Procedure}


\usepackage{tikz}
\usetikzlibrary{positioning,chains,fit,shapes,calc}

\definecolor{myblue}{RGB}{80,80,160}
\definecolor{mygreen}{RGB}{80,160,80}


\begin{document}

\title{FIN 373 Homework 3 \\ {\large due: \textbf{9/14/21}}}
\date{}
\maketitle

\vspace{-20mm}

\noindent Instructions: Please submit solutions on canvas.  Only a knitted pdf of an {\tt Rmarkdown} file will be accepted.
\\

\noindent \textbf{Problem 1:} QSS exercise 2.8.2.  In this exercise, we analyze the data from two experiments in which
households were canvassed for support on gay marriage.\footnote{This exercise is based on: LaCour, M. J., and
D. P. Green. 2014. \href{http://dx.doi.org/10.1126/science.1256151} {When Contact Changes Minds: An Experiment on Transmission of Support for Gay Equality.} \textit{Science} 346(6215): 1366-69.} Note that the original study was later retracted due to allegations of
fabricated data.  We will revisit this issue in the follow-up
exercise.  In this exercise, however, we analyze the original data
while ignoring the allegations.

Canvassers were given a script leading to conversations that averaged
about twenty minutes.  A distinctive feature of this study is that gay
and straight canvassers were randomly assigned to households and
canvassers revealed whether they were straight or gay in the course of
the conversation.  The experiment aims to test the 'contact
hypothesis,' which contends that out-group hostility (towards gays in
this case) diminishes when people from different groups interact with
one another.

The data file is {\tt gay.csv}. The names and descriptions of
variables are:
\vspace{3mm}
\begin{center}
\begin{tabular}{l p{12cm}}
 \hline
\textit{Variable} & \textit{Description} \\
\hline
\verb study     &         Study (1 or 2) \\
 
 \verb treatment &         Treatment assignment (5 possible options): {\tt No contact, Same-Sex Marriage Script by Gay Canvasser, Same-Sex Marriage Script by Straight Canvasser, Recycling Script by Gay Canvasser, and Recycling Script by Straight Canvasser} \\
 \verb wave       &        Survey wave (1-7). Note that Study 2 lacks wave 5 and 6. \\
 \verb ssm         &       Support for gay marriage (1 to 5).  
                      Higher scores indicate more support.\\ 
\hline
\end{tabular}
\end{center}
\vspace{2mm}

Each observation of this data set is a respondent giving a response to
a five-point survey item on same-sex marriage.  There are two
different studies in this data set, involving interviews during 7
different time periods (i.e. 7 waves).  In both studies, the first
wave consists of the interview before the canvassing treatment occurs.

\begin{enumerate}[a.]
	\item Using the baseline interview wave before the treatment is
  administered, examine whether randomization was properly conducted.
  Base your analysis on the three groups of Study 1: {\tt Same-Sex
  Marriage Script by Gay Canvasser, Same-Sex Marriage Script by
  Straight Canvasser and No Contact.}  Briefly comment on the
  results.
  \item The second wave of survey was implemented two months after the
  canvassing.  Using Study 1, estimate the average treatment effects
  of gay and straight canvassers on support for same-sex marriage,
  separately.  Give a brief interpretation of the results.
  \item The study contained another treatment that involves contact, but
  does not involve using the gay marriage script.  Specifically, the
  authors used a script to encourage people to recycle.  What is the
  purpose of this treatment?  Using Study 1 and wave 2, compare
  outcomes from the treatment {\tt Same-Sex Marriage Script by Gay
  Canvasser} to {\tt Recycling Script by Gay Canvasser.}  Repeat the
  same for straight canvassers, comparing the treatment {\tt Same-Sex
  Marriage Script by Straight Canvasser} to {\tt Recycling Script by
  Straight Canvasser.}  What do these comparisons reveal?  Give a
  substantive interpretation of the results.
  \item In Study 1, the authors reinterviewed the respondents 6
  different times (in waves 2 to 7) after treatment, at two month
  intervals.  The last interview in wave 7 occurs one year after
  treatment.  Do the effects of canvassing last?  If so, under what
  conditions?  Answer these questions by separately computing the
  average effects of straight and gay canvassers with the same-sex
  marriage script for each of the subsequent waves (relative to the
  control condition).
  \item The researchers conducted a second study to replicate the core
  results of the first study.  In this study, same-sex marriage
  scripts are only given by gay canvassers.  For Study 2, use the
  treatments {\tt Same-Sex Marriage Script by Gay Canvasser} and {\tt No
  Contact} to examine whether randomization was appropriately
  conducted.  Use the baseline support from wave 1 for this analysis.
  \item For Study 2, estimate the treatment effects of gay canvassing
  using data from wave 2.  Are the results consistent with those of
  Study 1?
  \item Using Study 2, estimate the average effect of gay canvassing at
  each subsequent wave and observe how it changes over time.  Note
  that Study 2 did not have 5th or 6th wave, but the 7th wave occurred
  one year after treatment as in Study 1.  Draw an overall conclusion
  from both Study 1 and Study 2.
\end{enumerate}

\vspace{7mm}
\noindent \textbf{Problem 2:} QSS exercise 2.8.3. One longstanding debate in the study of international relations
concerns the question of whether individual political leaders can make
a difference.  Some emphasize that leaders with different ideologies
and personalities can significantly affect the course of a nation.
Others argue that political leaders are severely constrained by
historical and institutional forces.  Did individuals like Hitler,
Mao, Roosevelt, and Churchill make a big difference?  The difficulty
of empirically testing these arguments stems from the fact that the
change of leadership is not random and there are many confounding
factors to be adjusted for.

In this exercise, we consider a \textit{natural experiment} in which the
success or failure of assassination attempts is assumed to be
essentially random.\footnote{This exercise is based on:
Jones, Benjamin F, and Benjamin A Olken. 2009. \href{http://dx.doi.org/10.1257/mac.1.2.55}{Hit or Miss? 
 The Effect of Assassinations on Institutions and 
 War.} \textit{American Economic Journal: Macroeconomics} 1(2): 55-87. }
 
 Each observation of the data set
{\tt leaders.csv} contains information about an assassination
attempt.  The variables are:
%\vspace{3mm}
\begin{center}
\begin{tabular}{l p{14.4cm}}
 \hline
\textit{Variable} & \textit{Description} \\
\hline
{\tt country} &           The name of the country \\
{\tt year} &               Year of assassination \\
{\tt leadername} &         Name of leader who was targeted \\
{\tt age}       &         Age of the targeted leader \\
{\tt politybefore} &       Average polity score during the 3 year period prior to the attempt \\
 {\tt polityafter} &        Average polity score during the 3 year period after the attempt\\
{\tt civilwarbefore} &     1 if country is in civil war during the 3 year period prior to the attempt, or 0 \\
 {\tt civilwarafter} &      1 if country is in civil war during the 3 year period after the attempt, or 0 \\
{\tt interwarbefore} &     1 if country is in international war during the 3 year period prior to the attempt, or 0 \\
 {\tt interwarafter} &      1 if country is in international war during the 3 year period after the attempt, or 0 \\
{\tt result} &             Result of the assassination attempt, one of 10 categories described below \\
\hline
\end{tabular}
\end{center}
\vspace{2mm}

The {\tt polity} variable represents the so-called \textit{polity score}
from the Polity Project.  The Polity Project systematically documents
and quantifies the regime types of all countries in the world from
1800.  The polity score is a 21-point scale ranging from -10
(hereditary monarchy) to 10 (consolidated democracy).  The {\tt result} variable is a 10 category factor variable describing
the result of each assassination attempt.

\begin{enumerate}[a.]
	\item How many assassination attempts are recorded in the data?  How
many countries experience at least one leader assassination attempt?
(The {\tt unique} function, which returns a set of unique values
from the input vector, may be useful here).  What is the average
number of such attempts (per year) among these countries?
	\item Create a new binary variable named {\tt success} that is equal
to 1 if a leader dies from the attack and to 0 if the leader
survives.  Store this new variable as part of the original data
frame.  What is the overall success rate of leader assassination?
Does the result speak to the validity of the assumption that the
success of assassination attempts is randomly determined?
	\item Investigate whether the average polity score over 3 years prior
  to an assassination attempt differs on average between successful
  and failed attempts.  Also, examine whether there is any difference
  in the age of targeted leaders between successful and failed
  attempts.  Briefly interpret the results in light of the validity of
  the aforementioned assumption.
  	\item Repeat the same analysis as in the previous question, but this
  time using the country's experience of civil and international war.
  Create a new binary variable in the data frame called
  {\tt warbefore}.  Code the variable such that it is equal to 1 if
  a country is in either civil or international war during the 3 years
  prior to an assassination attempt.  Provide a brief interpretation
  of the result.
  \item Does successful leader assassination cause democratization?
  Does successful leader assassination lead countries to war?  Answer
  these questions by analyzing the data.  Be sure to state your
  assumptions and provide a brief interpretation of the results.
\end{enumerate}

\end{document}

